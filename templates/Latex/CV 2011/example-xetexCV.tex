\documentclass{xetexCV}

\cvname{Albert Einstein}
\cvimage{Albert-Einstein-2.jpg}
\institution{Institute for Advanced Study}
\contactaddress{Einstein Drive\\
  Princeton, New Jersey \texttt{08540}\\
  United States of America}
\phonenumber{609-734-8000}
\faxnumber{609-924-8399}
\email{a.einstein@ias.edu}
\website{http://www.ias.edu/spfeatures/einstein/}

\hyphenpenalty=10000

% Set the Font to Warnock Pro and Frutiger LT Std
\setmainfont[Ligatures={Common}, Numbers={OldStyle}]{Warnock Pro}
\setsansfont{Frutiger LT Std}

\begin{document}
\makecvtitle

\cvsection{Biography}
Physicist and iconic genius of the 20th century.  Influenced most every touchstone of the modern era - the Bomb, the Big Bang, quantum physics and electronics.  Recognized in 1999 by Time Magazine as the ``Man of the Century''.

\cvsection{Current Appointment}
Professor Emeritus.  Institute for Advanced Study. Princeton.

\cvsection{Areas of Specialization}
Physics, Relativity Theory

\cvsection{Appointments Held}
University of Bern \years{1908-1911} \\
University of Zürich \years{1911-1912} \\
Charles University of Prague \years{1912-1914} \\
Prussian Academy of Sciences, Berlin \years{1914-1932} \\
University of Leiden \years{1920-1930} \\
Institute for Advanced Study, Princeton \years{1932-1955}

\cvsection{Education}
MSc in Physics, ETH Zurich \years{1900} \\
PhD in Physics, ETH Zurich \years{1900} 

\cvsection{Grants, Honors and Awards}
Nobel Prize in Physics, Nobel Foundation \years{1921} 

\pagebreak

\cvsection{Publications}
\cvsubsection{Journal Articles}
Einstein, Albert (1901) \years{1901}, “Folgerungen aus den Capillaritätserscheinungen (Conclusions Drawn from the Phenomena of Capillarity)", \emph{Annalen der Physik} 4: 513\medskip

Einstein, Albert (1905) \years{1901}, “On a Heuristic Viewpoint Concerning the Production and Transformation of Light", \emph{Annalen der Physik} 17: 132–148.\medskip

Einstein, Albert (1905) \years{1905b}, A new determination of molecular dimensions. \emph{PhD dissertation}.\medskip

Einstein, Albert (1905) \years{1905c}, “On the Motion—Required by the Molecular Kinetic Theory of Heat—of Small Particles Suspended in a Stationary Liquid", \emph{Annalen der Physik} 17: 549–560.\medskip

Einstein, Albert (1905) \years{1905d}, “On the Electrodynamics of Moving Bodies", \emph{Annalen der Physik} 17: 891–921.\medskip

Einstein, Albert (1905) \years{1905e}, “Does the Inertia of a Body Depend Upon Its Energy Content?", \emph{Annalen der Physik} 18: 639–641.\medskip

Einstein, Albert (1915) \years{1915}, “Die Feldgleichungen der Gravitation (The Field Equations of Gravitation)", \emph{Koniglich Preussische Akademie der Wissenschaften}: 844–847\medskip

Einstein, Albert (1917) \years{1917a}, “Kosmologische Betrachtungen zur allgemeinen Relativitätstheorie (Cosmological Considerations in the General Theory of Relativity)", \emph{Koniglich Preussische Akademie der Wissenschaften}\medskip

Einstein, Albert (1917) \years{1917b}, “Zur Quantentheorie der Strahlung (On the Quantum Mechanics of Radiation)", \emph{Physikalische Zeitschrift} 18: 121–128

\cvsubsection{Books}

Einstein, Albert (1954) \years{1954}, \emph{Ideas and Opinions}, New York: Random House\\
ISBN 0-517-00393-7

\cvsubsection{Newspaper Articles}

Einstein, Albert, et al. (December 4, 1948) \years{1940}, “To the editors", \emph{New York Times}\\
Einstein, Albert (May 1949) \years{1949}, “Why Socialism?", \emph{Monthly Review}.
\end{document}
