%
% Example for a Presentation with LaTeX
% Gerrit Hirschfeld
% Version: 02/2011
% Note: no R-code

%Latex-settings
\documentclass{beamer}					%If you want slides
%\documentclass[handout]{beamer}		%If you want handouts (= without animation) uncomment the next three lines 
%\usepackage{pgfpages}
%\pgfpagesuselayout{2 on 1}[a4paper,border shrink=5mm]
%\usepackage[ngerman]{babel}				%German Umlauts etc.
\usepackage[utf8]{inputenc}				%Mac
\usepackage{amsmath,amsfonts,amssymb}	%Symbols
\usepackage[hyper,notocbib]{apacite}				%Citations 
\bibliographystyle{apacite}				%APA-Style


%Beamer-settings
%\setbeamertemplate{navigation symbols}{}	%uncomment to remove the navigaton
\usetheme{Warsaw}						%a popular theme 
\usecolortheme{crane}					%yellowish-colors
\useinnertheme{rectangles}				%changes the header and footer
\useoutertheme{infolines}					%where to put the structure
\titlegraphic{\includegraphics[width=0.8\textwidth]{logo.jpg}}	%your logo


%
% The Document begins...
%
\begin{document}

%Title-slide
\title[Beamer-Example]{Some simple example slides generated with latex and the beamer class - Does not contain R-code-chunks, yet...}
\author[Dr. Hirschfeld]{Dr. rer. nat. Gerrit Hirschfeld, Dipl.-Psych.}
\institute[VIKP]{Vodafone Stiftungsinstitut für Kinderschmerztherapie und
Pädiatrische Palliativmedizin \\  Vestische Kinder und Jugendklinik Datteln }
\date{\today} 

\begin{frame}
\titlepage
\end{frame}

%Table Of Contents (TOC)
\begin{frame}\frametitle{}
	\tableofcontents
	%[pausesections]					%For an animated TOC
\end{frame} 


%Set-up the sections and subsections
\section{General Introduction} 
\subsection{Slides with bullets}

%Slide 1
\begin{frame}\frametitle{Slides with several blocks}
	non-block
	\begin{itemize}
	\item Item 1	
	\item Item 2
	\item Item 3
	\end{itemize}
	\begin{block}{Block 1: Bullets }
		\begin{itemize}
		\item Item 1
		\item Item 2
		\item Item 3
		\end{itemize}
	\end{block}
	\pause							%to "animate" the slides (everything after is only shown after a click)
	\begin{block}{Block 2 numbered}
		\begin{enumerate}
		\item Item 1
		\item Item 2
		\item Item 3
		\end{enumerate}
	\end{block}
\end{frame}

%More complex slide with columns and a graphic
\begin{frame}\frametitle{Results}
\begin{columns}
\begin{column}{0.6\textwidth}
	\begin{block}{behavior-only}
	Blabub \cite{Hirschfeld2010,Hirschfeld2011,Thielsch2010}.
	\end{block}
	\begin{block}{EEG}
	\cite{Hirschfeld2008}
	\end{block}
	\begin{block}{MEG}
	\cite{Hirschfeld2011a}
	\end{block}
\end{column}
\begin{column}{0.3\textwidth}
\begin{figure}[htbp] %  figure placement: here, top, bottom, or page
	\centering
	\includegraphics[width=\textwidth]{hirschfeld_2011}
	\caption{Hirschfeld et al. 2011}
\end{figure}
\end{column}
\end{columns}
\end{frame}

\begin{frame}\frametitle{}
\begin{columns}
\begin{column}{0.5\textwidth}
	\begin{block}{Thank you for your attention!}
	\end{block}
\end{column}
\end{columns}
\end{frame}

%more sections and subsections without slides
\section{Experiment 1}
\subsection{Method}
\subsection{Results}
\subsection{Discussion}
\section{Experiment 2}
\subsection{Method}
\subsection{Results}


\subsection{Discussion}
\section{General Diskussion}

%References
\begin{frame}[allowframebreaks]				%autmatically breaks the slides
\frametitle{References}
\bibliography{references}
\end{frame}


\end{document}