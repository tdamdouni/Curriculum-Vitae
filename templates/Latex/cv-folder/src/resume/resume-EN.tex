 \documentclass[11pt,a4paper]{moderncv}

\usepackage[T1]{fontenc}
\usepackage[utf8x]{inputenc}
\usepackage[swedish]{babel}
\usepackage[scale=0.8]{geometry}
\recomputelengths
\fancyfoot{}
\fancyfoot[LE,RO]{\thepage}
\fancyfoot[RE,LO]{\footnotesize }

% personal data
\name{John}{Doe}
\title{Resumé title}                               % optional, remove / comment the line if not wanted
\address{street and number}{postcode city}{country}% optional, remove / comment the line if not wanted; the "postcode city" and "country" arguments can be omitted or provided empty
\phone[mobile]{+1~(234)~567~890}                   % optional, remove / comment the line if not wanted; the optional "type" of the phone can be "mobile" (default), "fixed" or "fax"
\phone[fixed]{+2~(345)~678~901}
\phone[fax]{+3~(456)~789~012}
\email{john@doe.org}                               % optional, remove / comment the line if not wanted
\homepage{www.johndoe.com}                         % optional, remove / comment the line if not wanted
\social[linkedin]{john.doe}                        % optional, remove / comment the line if not wanted
\social[twitter]{jdoe}                             % optional, remove / comment the line if not wanted
\social[github]{jdoe}                              % optional, remove / comment the line if not wanted
\extrainfo{additional information}                 % optional, remove / comment the line if not wanted
\photo[84pt]{assets/images/cv_image.jpg}                      % optional, remove / comment the line if not wanted; '64pt' is the height the picture must be resized to, 0.4pt is the thickness of the frame around it (put it to 0pt for no frame) and 'picture' is the name of the picture file
\quote{"If you do what you've always done, you'll get what you've always gotten." -- Anthony Robbins}                 % optional, remove the line if not wanted

\moderncvtheme[black]{banking}

\begin{document}
%\begin{CJK*}{UTF8}{gbsn}                          % to typeset your resume in Chinese using CJK
%-----       resume       ---------------------------------------------------------
\makecvtitle

\section{Education}
\cventry{year--year}{Degree}{Institution}{City}{\textit{Grade}}{Description}  % arguments 3 to 6 can be left empty
\cventry{year--year}{Degree}{Institution}{City}{\textit{Grade}}{Description}

\section{Master thesis}
\cvitem{title}{\emph{Title}}
\cvitem{supervisors}{Supervisors}
\cvitem{description}{Short thesis abstract}

\section{Experience}
\subsection{Vocational}
\cventry{year--year}{Job title}{Employer}{City}{}{General description no longer than 1--2 lines.\newline{}%
Detailed achievements:%
\begin{itemize}%
\item Achievement 1;
\item Achievement 2, with sub-achievements:
  \begin{itemize}%
  \item Sub-achievement (a);
  \item Sub-achievement (b), with sub-sub-achievements (don't do this!);
    \begin{itemize}
    \item Sub-sub-achievement i;
    \item Sub-sub-achievement ii;
    \item Sub-sub-achievement iii;
    \end{itemize}
  \item Sub-achievement (c);
  \end{itemize}
\item Achievement 3.
\end{itemize}}
\cventry{year--year}{Job title}{Employer}{City}{}{Description line 1\newline{}Description line 2}
\subsection{Miscellaneous}
\cventry{year--year}{Job title}{Employer}{City}{}{Description}

\section{Languages}
\cvline{}
    {
        \small Self assessment according to
        \href{http://europass.cedefop.europa.eu/en/resources/european-language-levels-cefr}{CEFR}
        (C2 maximum evaluation\normalsize
    }
    \setlength{\tabcolsep}{5pt}
    \begin{tabular}{l r p{5mm}|  c c | c c | c}
            &  &   &       \multicolumn{2}{c}{\textbf{Understanding}} &            \multicolumn{2}{c}{\textbf{Speaking}}&
            \textbf{Writing} \\

            &
            &
            &
            Listening & Reading &
            Interaction & Production
            &\\

            \textbf{Swedish:}&
            Professional &
            &
            C2 &
            C2 &
            C2 &
            C2 &
            C2 \\

            \textbf{English:} &
            Experienced &
            &
            C2 &
            C2 &
            C2 &
            C2 &
            C2\\

            \textbf{Norweigan:}&
            Basic &
            &
            C1 &
            C1 &
            C1 &
            B2 &
            A1\\
    \end{tabular}

    \vspace{3mm}

    \closesection{}

\section{Computer skills}
\cvdoubleitem{category 1}{XXX, YYY, ZZZ}{category 4}{XXX, YYY, ZZZ}
\cvdoubleitem{category 2}{XXX, YYY, ZZZ}{category 5}{XXX, YYY, ZZZ}
\cvdoubleitem{category 3}{XXX, YYY, ZZZ}{category 6}{XXX, YYY, ZZZ}

\section{Interests}
\cvitem{hobby 1}{Description}
\cvitem{hobby 2}{Description}
\cvitem{hobby 3}{Description}

\section{Extra 1}
\cvlistitem{Item 1}
\cvlistitem{Item 2}
\cvlistitem{Item 3. This item is particularly long and therefore normally spans over several lines. Did you notice the indentation when the line wraps?}

\section{Extra 2}
\cvlistdoubleitem{Item 1}{Item 4}
\cvlistdoubleitem{Item 2}{Item 5\cite{book1}}
\cvlistdoubleitem{Item 3}{Item 6. Like item 3 in the single column list before, this item is particularly long to wrap over several lines.}

% Publications from a BibTeX file without multibib
%  for numerical labels: \renewcommand{\bibliographyitemlabel}{\@biblabel{\arabic{enumiv}}}% CONSIDER MERGING WITH PREAMBLE PART
%  to redefine the heading string ("Publications"): \renewcommand{\refname}{Articles}
\nocite{*}
\bibliographystyle{plain}
\bibliography{publications}                        % 'publications' is the name of a BibTeX file

% Publications from a BibTeX file using the multibib package
%\section{Publications}
%\nocitebook{book1,book2}
%\bibliographystylebook{plain}
%\bibliographybook{publications}                   % 'publications' is the name of a BibTeX file
%\nocitemisc{misc1,misc2,misc3}
%\bibliographystylemisc{plain}
%\bibliographymisc{publications}                   % 'publications' is the name of a BibTeX file

\clearpage
\end{document}
