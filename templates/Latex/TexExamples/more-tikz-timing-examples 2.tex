% TikZ-Timing v0.7 2009/12/09 Example 8 & 9:
% Author: Martin Scharrer
%
% This is an example for the 'tikz-timing' package.
% It redraws the timing diagram taken from
% http://commons.wikimedia.org/wiki/File:SR_FF_timing_diagram.png
% http://en.wikipedia.org/wiki/File:SPI_timing_diagram.svg
%
\documentclass{article}
%%%<
\usepackage{verbatim}
%%%>
\begin{comment}
:Title: More tikz-timing examples
:Grid: 1x2

This example shows how to make a timing diagrams with the `tikz-timing <http://www.ctan.org/tex-archive/help/Catalogue/entries/tikz-timing.html>`_  package. The timing diagrams were taken from an external source and redrawn in TeX to demonstrate the necessary steps.

The external sources are:

- `<http://commons.wikimedia.org/wiki/File:SR_FF_timing_diagram.png>`_
- `<http://en.wikipedia.org/wiki/File:SPI_timing_diagram.svg>`_

Timing diagrams like these can be done using the tikztimingtable environment which has the same syntax as a tabular environment with two columns. The first column holds the signal name, the second one the timing characters. See the `package manual <http://www.ctan.org/tex-archive/graphics/pgf/contrib/tikz-timing/tikz-timing.pdf>`_ for detailed information about them.

\end{comment}

\usepackage{tikz-timing}[2009/12/09]
% Use tikz-timing library 'counters' to define a counter character.
% This character draws a 'D{<counter value>}' and increases the counter
% value by one. A reset character which resets the counter value 
% (by default to 1) is also defined.
\usetikztiminglibrary[new={char=Q,reset char=R}]{counters}
\usepackage[active,tightpage]{preview}
\PreviewEnvironment{tikzpicture}
\setlength{\PreviewBorder}{5mm}
\pagestyle{empty}

\begin{document}
%
% Defining foreground (fg) and background (bg) colors
\definecolor{bgblue}{rgb}{0.41961,0.80784,0.80784}%
\definecolor{bgred}{rgb}{1,0.61569,0.61569}%
\definecolor{fgblue}{rgb}{0,0,0.6}%
\definecolor{fgred}{rgb}{0.6,0,0}%
%
\begin{tikztimingtable}[
    timing/slope=0,         % no slope
    timing/coldist=2pt,     % column distance
    xscale=2.05,yscale=1.1, % scale diagrams
    semithick               % set line width
  ]
  \scriptsize clock     & 7{C}                              \\
  S                     & [fgblue] .75L h 2.25L H LLl       \\
  R                     & [fgblue]  1.8L .8H 2.2L 1.4H 0.8L \\
  Q                     &          L .8H 1.7L 1.5H LL       \\
  $\overline{\mbox{Q}}$ &          H .8L 1.7H 1.5L HH       \\
  Q                     & [fgred]  HLHHHLL                  \\
  $\overline{\mbox{Q}}$ & [fgred]  LHLLLHH                  \\
\extracode
 \makeatletter
 \begin{pgfonlayer}{background}
  % Draw shaded backgrounds
  \shade [right color=bgblue,left color=white]
     (7,-8.45) rectangle (-2,-4.6);
  \shade [right color=bgred,left color=white]
     (7,-12.8) rectangle (-2,-8.6);
  % Add background grid lines
  \begin{scope}[gray,semitransparent,semithick]
    \horlines{}
    \foreach \x in {1,...,6}
      \draw (\x,1) -- (\x,-12.8);
    % similar: \vertlines{1,...,6}
  \end{scope}
  % Add labels
  \node [anchor=south east,inner sep=0pt]
    at (7,-8.45) {\tiny clocked};
  \node [anchor=south east,inner sep=0pt,fgred]
    at (7,-12.8) {\tiny positive edge triggered};
 \end{pgfonlayer}
\end{tikztimingtable}%
%
\begin{tikztimingtable}
  [timing/d/background/.style={fill=white},
   timing/lslope=0.2]
          CPOL=0 & LL 15{T} LL \\
          CPOL=1 & HH 15{T} HH \\
                 & H 17L H     \\
  \\
        Cycle \# & U     R 8{2Q} 2U    \\
            MISO & D{z}  R 8{2Q} 2D{z} \\
            MOSI & D{z}  R 8{2Q} 2D{z} \\
  \\
        Cycle \# & UU    R 8{2Q} U    \\
            MISO & D{z}U R 8{2Q} D{z} \\
            MOSI & D{z}U R 8{2Q} D{z} \\
\extracode
  % Add vertical lines in two colors
  \begin{pgfonlayer}{background}
    \begin{scope}[semitransparent,semithick]
      \vertlines[red]{2.1,4.1,...,17.1}
      \vertlines[blue]{3.1,5.1,...,17.1}
    \end{scope}
  \end{pgfonlayer}
  % Add big group labels
  \begin{scope}
    [font=\sffamily\Large,shift={(-6em,-0.5)},anchor=east]
    \node at (  0, 0) {SCK};    \node at (  0,-3 ) {SS};
    \node at (1ex,-9) {CPHA=0}; \node at (1ex,-17) {CPHA=1};
  \end{scope}
\end{tikztimingtable}%
\end{document}
