\section{\Index{Hollowing out} a solid's faces}

We call \textit{hollowing by the ratio $k$} an operation, which for a given
face with the center $G$, executes a dilation on that face with the ratio
$k$, then divides the original face with using this new face.

For example, a cube with a hollow of its top face with a ratio of $0.8$:

\begin{center}
\psset{unit=0.5}
\psset{lightsrc=10 0 10,viewpoint=50 -20 30 rtp2xyz,Decran=50}
\begin{pspicture*}(-4,-4)(4,4)
%\psframe(-4,-4)(4,4)
\psSolid[object=cube,
   fillcolor=red,
   affinagerm,
   fcolor=Yellow,
   affinage=0]
\end{pspicture*}
\end{center}

The option \Lkeyword{affinage} allows us to hollow a solid's faces either globally or
individually. This option uses the key \Lkeyword{affinagecoeff}
(value $0.8$ by default) which indicates the ratio $k$ used for the
hollow ($0<k<1$).
%
\begin{compactitem}
 \item \texttt{\Lkeyword{affinage}=\Lkeyval{all}} hollows all the faces;
 \item \texttt{\Lkeyword{affinage}=0 1 2 3} hollows the faces 0, 1, 2 and 3;
\end{compactitem}

When a face is hollowed out, the default behaviour suppresses the resulting central
face. However, the option \Lkeyword{affinagerm} allows us to conserve that central face.

When we conserve the centre face, it is---by default---drawn with the same colour
as the original. The option \Lkeyword{fcolor} permits to specify another colour.

%\newpage
\psset{lightsrc=10 0 10,viewpoint=50 -20 30 rtp2xyz,Decran=50}
\begin{LTXexample}[width=6cm]
\psset{unit=0.5}
\begin{pspicture*}(-5,-4)(6,5)
\psSolid[object=cube,
   fillcolor=cyan,
   incolor=red,
   hollow,
   affinage=0]
\end{pspicture*}
\end{LTXexample}
%

\psset{lightsrc=10 0 10,viewpoint=50 -20 30 rtp2xyz,Decran=50}
\begin{LTXexample}[width=6cm]
\psset{unit=0.5}
\begin{pspicture*}(-5,-4)(6,5)
\psSolid[object=cube,
   fillcolor=cyan,
   affinagecoeff=.5,
   affinagerm,
   fcolor=.5 setfillopacity Yellow,
   hollow,
   affinage=all]
\end{pspicture*}
\end{LTXexample}

\endinput
