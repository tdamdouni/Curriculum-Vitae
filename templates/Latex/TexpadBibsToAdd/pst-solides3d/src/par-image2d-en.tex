\section{Projection of images}


This command displays an eps image on a plane defined by an origin and a normal, this plan can be the face 
of a predefined object: a  cube for example. The eps image must be prepared according to the method 
described in the documentation for 
`\textsf{pst-anamorphosis'}\footnote{\url{http://melusine.eu.org/syracuse/G/pst-anamorphosis/doc/}}.

The macro includes various options:
\begin{verbatim}
    \psImage[file=<filename with extension>,
             divisions=10,
             normale=nx ny nz,
             origine=xO yO zO,
             phi=angle,
             unitPicture=28.45](x,y)
\end{verbatim}
It focuses the image on the plane at the point defined by the origin, it may be moved to another point 
by setting the \emph{optional} values \verb+(x,y)+. You can omit these values 
if we do not translate the image into another point than the origin of the plan.

\psframebox[linestyle=none,fillcolor=yellow,fillstyle=solid]{\texttt{divisions=20}} 
selects the number of sub-segments for \texttt{lineto} in the image file to display. The higher the number, 
the higher the projected image will be faithful to the original. However, the projection takes place on a 
plane, the deformation will be small in all cases except one approaches very close to the plane, therefore 
a small number of sub-divisions will generally give a correct result and will perform calculations quickly .

\psframebox[linestyle=none,fillcolor=yellow,fillstyle=solid]{\texttt{phi}} can rotate the image of a fixed 
value in degrees.

\psframebox[linestyle=none,fillcolor=yellow,fillstyle=solid]{\texttt{unitImage=28.45}} 
allows to resize the size of the eps image that is generally points per cm, a larger value will give a smaller image.

If you want to place the image on the front of an object, it will follow the following procedure:
\begin{itemize}
   \item determine the number of faces of the object, see the documentation of `\textsf{pst-solides3d} ';
   \item give to the normal of the face in question and origin at the center of that face. We can always 
   shift the image with \verb+(x, y)+.
\end{itemize}

\begin{verbatim}
\begin{pspicture}(-5,-5)(5,5)
\psset{solidmemory}
\psSolid[object=cube,a=8,action=draw,name=OBJECT,linecolor=red]%
\psImage[file=tiger.eps,normal=OBJECT 0 solidnormaleface,
         origine=OBJECT 0 solidcentreface,unitPicture=75]
\psImage[file=tiger.eps,normal=OBJECT 1 solidnormaleface,
         origine=OBJECT 1 solidcentreface,unitPicture=75]
\psImage[file=tiger.eps,normal=OBJECT 4 solidnormaleface,
         origine=OBJECT 4 solidcentreface,unitPicture=75]
\psImage[file=tiger.eps,normal=OBJECT 3 solidnormaleface,
         origine=OBJECT 3 solidcentreface,unitPicture=75]
\psImage[file=tiger.eps,normal=OBJECT 2 solidnormaleface,
         origine=OBJECT 2 solidcentreface,unitPicture=75]
\end{pspicture}
\end{verbatim}

If the selected plan is not visible to the set position, it may, if desired, force the display of the 
image with the \verb+visibility+.



\begin{pspicture}(-10,-4)(6,13)
\psframe(-10,-4)(6,13)
\psset{viewpoint=12 60 20 rtp2xyz,Decran=10,lightsrc=viewpoint}
\psImage[file=tiger.eps,normal=1 0 0,origine=0 2 2](0,3)
\psSolid[object=plan,
         definition=normalpoint,
         args={0.01 2 2 [1 0 0 90]},
         action=draw,linecolor=red,
         planmarks,
         showBase,
         base=-2 2 -2 4]
\psImage[file=tiger.eps,normal=0 1 0,origine=2 0 2]%(0,0)
\psSolid[object=plan,
         definition=normalpoint,
         args={2 0.01 2 [0 1 0 180]},
         action=draw,linecolor=red,
         planmarks,
         showBase,
         base=-2 2 -2 2]
\psImage[file=tiger.eps,normal=0 0 1,origine=2 2 0](2,0)
\psSolid[object=plan,
         definition=normalpoint,
         args={2 2 0.01 [0 0 1 90]},
         action=draw,linecolor=red,
         planmarks,
         showBase,
         base=-2 3 -2 2]%
\psImage[file=parrot.eps,normal=1 1 1,origine=5 5 5,unitPicture=75,phi=90]%(0,0)
\psSolid[object=plan,
         definition=normalpoint,
         args={5 5 5 [1 1 1 180]},
         action=draw,linecolor=red,
         planmarks,
         showBase,
         base=-2 2 -2 2]
\axesIIID(0,0,0)(4,5,6)
\end{pspicture}


\psset{unit=1cm}
\psset{viewpoint=20 -120 30 rtp2xyz,Decran=20,unitPicture=15,lightsrc=viewpoint}
\begin{pspicture}[solidmemory](-5,-7.5)(7,6)
\psSolid[object=cube,a=8,name=OBJECT,linecolor=red,fillcolor=white%,numfaces=all,fontsize=100
]
\psset[pst-solides3d]{normal=OBJECT 0 solidnormaleface}
\psImage[file=tiger.eps,origine=OBJECT 0 solidcentreface,phi=-90](0,0)
%\psset[pst-solides3d]{normal=OBJECT 1 solidnormaleface}
%\psImage[file=tiger.eps,
%  origine=OBJECT 1 solidcentreface]
%\psset[pst-solides3d]{normal=OBJECT 4 solidnormaleface}
%\psImage[file=tiger.eps,origine=OBJECT 4 solidcentreface]
\psset[pst-solides3d]{normal=OBJECT 3 solidnormaleface}
\psImage[file=tiger.eps,origine=OBJECT 3 solidcentreface]
\psset[pst-solides3d]{normal=OBJECT 2 solidnormaleface}
\psImage[file=tiger.eps,origine=OBJECT 2 solidcentreface]
\end{pspicture}



\section{A bit of theory}

\begin{minipage}{.45\textwidth}
The image is projected into a plane defined by a normal $\vec{K}$ and origin $O'(x_O,y_O,z_O)$. 
The coordinates of points in each image are given in reference to a benchmark plan 
$(O,\vec {I},\vec{J})$ whose vectors are determined from $\vec{K}$ as follows:
This vector $\vec{K}$ is defined by $\theta$ and $\varphi$, we calculate these values from the coordinates.
With $(O,\vec{i},\vec{j},\vec{k})$

\begin{align*}
 \vec{K}=\left(
 \begin{aligned}
 \cos\varphi & \cos\theta\\
 \cos\varphi & \sin\theta\\
 \sin\varphi
\end{aligned}% 
\right)
\end{align*}

You must then choose the other two basis vectors
 $(\vec{I},\vec{J},\vec{K})$.
I choose to keep $\vec{I}$ at the plane $Oxy$
\end{minipage}
%
\hfill
%
\begin{minipage}{0.45\textwidth}
\begin{pspicture}(-3,-5)(4,5)
\psset{unit=5}
\psset{viewpoint=50 15 20  rtp2xyz ,Decran=35}
\psset{solidmemory}
\pstVerb{/Theta 45 def /Phi 45 def
         /cosPhi {Phi cos} bind def
         /sinPhi {Phi sin} bind def
         /cosTheta {Theta cos} bind def
         /sinTheta {Theta sin} bind def
         /Kx {cosPhi cosTheta mul} bind def
         /Ky {cosPhi sinTheta mul} bind def
         /Kz sinPhi def}%
\psSolid[object=plan,definition=normalpoint,args={0 0 0 [Kx Ky Kz 145]},action=draw,linecolor=blue,base=-1 1 -1 0]
\psSolid[object=plan,definition=normalpoint,args={0 0 0 [0 0 1]},action=draw**,linecolor=red,base=-1 1 -1 1]
\axesIIID(0,0,0)(1.25,1.25,1.25)
\psSolid[object=plan,definition=normalpoint,args={0 0 0 [0 0 1]},action=none,linecolor=red,name=Oxy,base=-1 1 -1 1]
\psset{plan=Oxy}%
\psProjection[object=cercle,resolution=360,args=0 0 1,linecolor=gray,linestyle=dashed,range=0 360]
\psProjection[object=texte,text=q,fontsize=5,PSfont=Symbol,isolatin=false,phi=90](.25,0.125)%
\psSolid[object=plan,
         definition=normalpoint,
         args={0 0 0 [1 -1 0]},
         action=none,linecolor=red,
         name=Oxz,
         base=-1 1 -1 1
         ]%
\psset{plan=Oxz}%
\psProjection[object=cercle,resolution=360,
              args=0 0 1,linecolor=gray,
              linestyle=dashed,
              range=0 90]%
\psSolid[object=vecteur,
        definition={[.02 .1]},
        linecolor={[cmyk]{1,0,1,0.5}},
        args=Kx Ky Kz](0,0,0)
\psSolid[object=plan,
         definition=normalpoint,
         args={0 0 0 [Kx Ky Kz 145]},
         action=draw,linecolor=blue,
         base=-1 1 0 1,name=projection]%
\psProjection[object=texte,plan=projection,text=plan de projection,fontsize=4](0,0.85)%
\psSolid[object=vecteur,
        definition={[.02 .1]},
        linecolor=red,
        args=sinTheta neg cosTheta 0 ](0,0,0)%
\psSolid[object=plan,
         definition=normalpoint,
         args={0 0 0 [sinTheta cosTheta neg 0]},
         action=none,linecolor=blue,
         base=-1 1 -1 1,name=verticale]%
\psProjection[object=texte,text=f,plan=verticale,PSfont=Symbol,isolatin=false,fontsize=4](0.5,0.2)%
\psSolid[object=vecteur,
        definition={[.02 .1]},
        linecolor=blue,
        args=cosTheta neg sinPhi mul sinTheta neg sinTheta mul cosPhi ](0,0,0)%
\psPoint(Kx, Ky,Kz){K}
\psPoint(Kx, Ky,0){XY}
\psPoint(Kx,0,0){X}
\psPoint(0, Ky,0){Y}
\psPoint(0,0,0){O}
\psPoint(cosTheta neg sinPhi mul, sinTheta neg sinTheta mul, cosPhi){J}
\psPoint(sinTheta neg, cosTheta, 0){I}
\psline(O)(XY)
\psline[linestyle=dashed](XY)(K)
\psline(X)(XY)(Y)
\pstVerb{/xTube {t Cos 0.4 mul} def /yTube {t Sin 0.4 mul} def /zTube {0} def}%
\defFunction{F}(t){xTube}{yTube}{zTube}%
% choix de deux points très voisins sur le tube
\pstVerb{/t1 0.22 pi mul def /t2 0.25 pi mul def }%
\psPoint(/t t1 def xTube ,yTube,zTube){A}
\psPoint(/t t2 def xTube ,yTube,zTube){B}
\psSolid[object=courbe,
   r=0,
   function=F,
   range=0 0.25 pi mul,
   fillcolor=red]
\psline[linecolor=red,arrowsize=0.03]{->}(A)(B)
%
\pstVerb{/xT {t Cos 0.4 mul cosPhi mul} def /yT {t Cos 0.4 mul cosPhi mul} def /zT {t Sin 0.4 mul} def}%
\defFunction{F}(t){xT}{yT}{zT}%
% choix de deux points très voisins sur le tube
\pstVerb{/t1 0.22 pi mul def /t2 0.25 pi mul def }%
\psPoint(/t t1 def xT ,yT,zT){A}
\psPoint(/t t2 def xT ,yT,zT){B}
\psSolid[object=courbe,
   r=0,
   function=F,
   range=0 0.25 pi mul,
   fillcolor={[rgb]{0.3,0.18,0.18}}]
\psline[linecolor={[rgb]{0.3,0.18,0.18}},arrowsize=0.03]{->}(A)(B)
\uput[u](J){\blue$\overrightarrow{J}$}
\uput[ur](K){\color[cmyk]{1,0,1,0.5}{$\overrightarrow{K}$}}
\uput[r](I){\red$\overrightarrow{I}$}    %$
\end{pspicture}
\end{minipage}

\endinput



Seen from above, in the plane $Oxy$:

\begin{minipage}{.4\textwidth}
\[
\overrightarrow{I}=\left(%
 \begin{aligned}
 -\sin\theta\\
 \hphantom{-}\cos\theta\\
 0
 \end{aligned}
 \right)
 \]
\end{minipage}
\hfill
\begin{minipage}{0.5\textwidth}

\begin{pspicture}(-3,-4)(4,2)
\psline{->}(4,0)\uput[0](4,0){$y$}
\psline[linestyle=dashed](0,2)
\psline{->}(0,-3.5)\uput[270](0,-3.5){$x$}
\uput[135](0,0){O}
{\psset{linewidth=2\pslinewidth}
\psline{->}(0,-2)\uput[0](0,-2){$\overrightarrow{i}$}
\psline{->}(2,0)\uput[90](2,0){$\overrightarrow{j}$}
\psline[linestyle=dotted](3;-30)\uput[0](3;-30){$x'$}
\psline[linecolor=red,doubleline=true]{->}(2;60)\uput[0](2;60){$\red \overrightarrow{I}$}
}
\psarc{->}(0,0){1.5}{-90}{-30}\uput[0](1.6;-60){$\theta$}
\end{pspicture}
\end{minipage}

Il reste à trouver $\overrightarrow{J}$ pour que la base
($\overrightarrow{I},\overrightarrow{J},\overrightarrow{K}$) soit directe :
$\overrightarrow{J}=\overrightarrow{K}\times\overrightarrow{I}$
\[
\overrightarrow{J}=\left(\begin{aligned}
 \cos\varphi\cos\theta\\
 \cos\varphi\sin\theta\\
 \sin\varphi
 \end{aligned}
 \right)
\times
\left(
 \begin{aligned}{c}
 -\sin\theta\\
 \hphantom{-}\cos\theta\\
 0
 \end{aligned}
 \right)
 =
 \left(\begin{aligned}
 -\sin\varphi\cos\theta\\
 -\sin\varphi\sin\theta\\
 \cos\varphi
 \end{aligned}
 \right)
 \]
 The transformation matrice:
\[
A=\left(%
 \begin{array}{ccc}
 -\sin\theta&-\sin\varphi\cos\theta&\cos\varphi\cos\theta\\
 \hphantom{-}\cos\theta&-\sin\varphi\sin\theta&\cos\varphi\sin\theta\\
 0&\cos\varphi&\sin\varphi
 \end{array}
 \right)
 \]

to determine the coordinates ($ x, y, z $) of a point $M$ if one knows its
  coordinates $(X, Y, Z)$ in the reference
 $(O,\overrightarrow{I},\overrightarrow{J},\overrightarrow{K})$.

 \[
 \left(\begin{aligned}{c}
 x\\
 y\\
 z
 \end{aligned}
 \right)
 =\left(%
 \begin{array}{ccc}
 -\sin\theta&-\sin\varphi\cos\theta&\cos\varphi\cos\theta\\
 \hphantom{-}\cos\theta&-\sin\varphi\sin\theta&\cos\varphi\sin\theta\\
 0&\cos\varphi&\sin\varphi
 \end{array}
 \right)
\left(\begin{aligned}
 X\\
 Y\\
 Z
 \end{aligned}
 \right)
\]
\[
\left\lbrace\begin{array}{cccclcl}
x&=&-X\sin\theta&-&Y\sin\varphi\cos\theta&+&Z\cos\varphi\cos\theta\\
y&=&\hphantom{-}X\cos\theta&-&Y\sin\varphi\sin\theta&+&Z\cos\varphi\sin\theta\\
z&=&0&+&Y\cos\varphi&+&Z\sin\varphi
\end{array}
\right.
\]

If we consider a point on the plane in the plane $XOY$

\[
\left\lbrace\begin{array}{ccccl}
x&=&-X\sin\theta&-&Y\sin\varphi\cos\theta\\
y&=&\hphantom{-}X\cos\theta&-&Y\sin\varphi\sin\theta\\
z&=&0&+&Y\cos\varphi
\end{array}
\right.
\]
Et si maintenant, ce repère $OXYZ$ est translaté en un point
$O'(x_{O'},y_{O'},z_{O'})$
\[
\left\lbrace\begin{array}{cccclcl}
x&=&-X\sin\theta&-&Y\sin\varphi\cos\theta&+&x_{O'}\\
y&=&\hphantom{-}X\cos\theta&-&Y\sin\varphi\sin\theta&+&y_{O'}\\
z&=&0&+&Y\cos\varphi&+&z_{O'}
\end{array}
\right.
\]

Remarks:
\begin{itemize}
\item  $\overrightarrow{K}$ since we can obviously choose another associated base $ (\overrightarrow {I}, 
\overrightarrow {J}) $ by turning the previously calculated around $ \overrightarrow {K} $ of the selected angle. 
For this first draft order I preferred to rotate the image, which probably has the disadvantage lengthen calculations \ldots\

\item Jean-Paul Vigneault made a different choice for the base $(\overrightarrow{I}, \overrightarrow{J} $, 
he calculated $ \overrightarrow {J}$ from $\overrightarrow {K} $ by relation:
\[
\overrightarrow{J}=\overrightarrow{K}\wedge \left(%
 \begin{array}{c}
 1\\
 0\\
 0
 \end{array}
 \right)
\]
 $\overrightarrow{I}=\overrightarrow{J}\wedge\overrightarrow{K}$.
We can bring the system defined in `\textsf{pst-solides3d}' to the one I chose by setting the \textsf{phi} 
of `\textsf{pst-solides3d}' (which allows you to turn the mark ) the proper value \ldots\ to calculate.

\end{itemize}

