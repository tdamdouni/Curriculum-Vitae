
% Minimalbeispiel 
\documentclass[a4paper,fontsize=11pt,parskip=half]{scrartcl} 
\usepackage[T1]{fontenc} 
\usepackage[utf8]{inputenc} 
\usepackage[ngerman]{babel} 
\usepackage[demo]{graphicx} 
\usepackage{picinpar,wrapfig,fancyvrb,lipsum,bera} 

\addtokomafont{caption}{\itshape} 
\addtokomafont{captionlabel}{\sffamily\bfseries} 
\renewcommand*{\captionformat}{~--~} 

\begin{document} 
\section*{Minimalbeispiel} 
Ein bisschen test Text der nicht von blindtext generiert wurde da das 
glaube ich einen kleinen Unterschied macht wo man denn letztendlich 
dann die wrapfigure Umgebung platziert. 

\begin{figure}[h!] 
\centering 
\includegraphics[width=4cm,height=2cm]{plank} 
\caption{Blick in die perfekte Nacht} 
\end{figure} 

\begin{figwindow}[2,l,{ 
\unitlength15mm 
\begin{picture}(3,1.4) 
\put(0.7,0.7){\circle*{0.2}} \put(0.7,0.7){\circle{1.2}} 
\put(0.7,0.7){\vector(0,1){0.6}} \put(2.5,0.7){\circle*{0.5}} 
\end{picture}},{Kreise}] 
\lipsum*[3] 
\end{figwindow} 

\begin{wrapfigure}[9]{l}{6cm}% 
\vspace{-0.8\baselineskip} 
\centering% 
\rule{6cm}{3.2cm}% 
\caption{Mit wrapfigure} 
\end{wrapfigure}% 
Test\footnote{Da stolpert manchmal der Algorithmus.} -- \lipsum*[6-8] 

% Das geht auch alles mit \parshape 
\vspace{\parskip} 
\hfill\smash{% Adjustment for height/depth 
\raisebox{\dimexpr-\height-1.5\baselineskip}{% 
\begin{minipage}{6.0cm}% 
\includegraphics[width=6.0cm,height=2cm]{blank}% 
\captionof{figure}{Mit \textbackslash parshape}% 
\end{minipage}}% 
}\\[\dimexpr-2\baselineskip-2\parskip] 

% \parshape <num lines> <indent> <width> ... 
\parshape 11 
0pt \textwidth 0pt \textwidth 
0pt 0.55\textwidth 0pt 0.55\textwidth 
0pt 0.55\textwidth 0pt 0.55\textwidth 
0pt 0.55\textwidth 0pt 0.55\textwidth 0pt 0.55\textwidth 
0pt \textwidth 0pt \textwidth 
Mit etwas mehr Handarbeit geht das jetzt auch anders mit 
\texttt{\textbackslash parshape}.\footnote{Damit braucht man 
dann auch kein picinpar mehr.} \lipsum[1] 
\parshape 0 

\subsection*{Wer ist das Problem, picinpar oder scrartcl?} 
Wie man leicht sieht, gibt es hier zwei Grafiken mit \texttt{caption}, 
dessen Formatierung folgendermaßen angepasst wurde. 

\begin{Verbatim}[fontsize=\footnotesize] 
\addtokomafont{caption}{\itshape} 
\addtokomafont{captionlabel}{\sffamily\bfseries} 
\renewcommand*{\captionformat}{~--~} 
\end{Verbatim} 

Jedoch kümmert sich die textumflossene Grafik \texttt{Kreise} -- die mit 
dem Paket \texttt{picinpar} eingebracht wurde -- überhaupt nicht darum. 
\end{document} 
