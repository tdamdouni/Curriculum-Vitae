
% Minimalbeispiel 
\documentclass[a4paper,fontsize=12pt,parskip=half]{scrartcl} 
\usepackage[T1]{fontenc} 
\usepackage[utf8]{inputenc} 
\usepackage[ngerman]{babel} 
\usepackage[demo]{graphicx} 
\usepackage{bera,picinpar,fancyvrb,lipsum} 

\addtokomafont{caption}{\itshape} 
\addtokomafont{captionlabel}{\sffamily\bfseries} 
\renewcommand*{\captionformat}{~--~} 

\begin{document} 
\section*{Minimalbeispiel} 
\lipsum*[2] 

\begin{figure}[h!] 
\centering 
\includegraphics[width=4cm,height=2cm]{plank} 
\caption{Blick in die perfekte Nacht} 
\end{figure} 

\begin{figwindow}[2,l,{ 
\unitlength15mm 
\begin{picture}(3,1.4) 
\put(0.7,0.7){\circle*{0.2}} \put(0.7,0.7){\circle{1.2}} 
\put(0.7,0.7){\vector(0,1){0.6}} \put(2.5,0.7){\circle*{0.5}} 
\end{picture}},{Kreise}] 
\lipsum*[3] 
\end{figwindow} 

\subsection*{Wer ist das Problem, picinpar oder scrartcl?} 
Wie man leicht sieht, gibt es hier zwei Grafiken mit \texttt{caption}, 
dessen Formatierung folgendermaßen angepasst wurde. 

\begin{Verbatim}[fontsize=\footnotesize] 
\addtokomafont{caption}{\itshape} 
\addtokomafont{captionlabel}{\sffamily\bfseries} 
\renewcommand*{\captionformat}{~--~} 
\end{Verbatim} 

Jedoch kümmert sich die textumflossene Grafik -- die mit dem Paket 
\texttt{picinpar} eingebracht wurde -- überhaupt nicht darum. 
\end{document} 