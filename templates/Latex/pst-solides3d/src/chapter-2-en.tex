\section{Choice of the view point}

\begin{center}

\begin{pspicture}(-5,-5.7)(10,7)
\psset{lightsrc=10 20 30,viewpoint=50 30 20 rtp2xyz}
\definecolor{bleuciel}{rgb}{0.78,0.84,0.99}
\psSolid[object=cube,fillcolor=bleuciel,a=2,action=draw*]%%
%\psSolid[object=cubemaillage,fillcolor=bleuciel,a=2]%%
\psSolid[object=grille,base=0 8 0 10,action=draw]%%
\psSolid[object=grille,base=0 7 0 10,action=draw,RotY=90](0,0,7)%
\psSolid[object=grille,base=0 8 0 7,action=draw,RotX=-90](0,0,7)%
\psSolid[object=cube,fillcolor=bleuciel,a=1,action=draw*](0.5,0.5,0.5)%
\psSolid[object=grille,base=-1 1 -1 1,action=draw,linecolor=blue](0,0,1)%
\psSolid[object=grille,base=-1 1 -1 1,action=draw,RotY=90,linecolor=blue](1,0,0)%
\psSolid[object=grille,base=-1 1 -1 1,action=draw,RotX=-90,linecolor=blue](0,1,0)%
\axesIIID(1,1,1)(8,10,7)
\pstVerb{
   /dV 12 def % distance V
   /dE 6 def % distance \'{e}cran
   /Theta 60 def
   /Phi 30 def
   dV Theta Phi rtp2xyz
   /zV exch def
   /yV exch def
   /xV exch def
   dE Theta Phi rtp2xyz
   /zE exch def
   /yE exch def
   /xE exch def
         }%
\psPoint(xV,yV,zV){V}
\psPoint(xE,yE,zE){E}
\psPoint(xV,yV,0){Vp}
%
% 5 distance \'{e}cran
%\psPoint(dE Theta  cos mul Phi cos div dE Theta  sin mul Phi cos div 0){Vq}
\psPoint(xV,0,0){Vx}
\psPoint(0,yV,0){Vy}
\psPoint(0,0,zV){Vz}
\psdot(V)
{\psset{linestyle=dashed,linecolor=red}
\psline(V)(Vp)\psline(Vx)(Vp)\psline(Vy)(Vp)\psline(V)(Vz)\psline(V)(O)\psline(Vp)(O)}
%\psSolid[object=grille,base=-5 5 -3 3,action=draw,RotX=-60,linecolor=red](xE,yE,zE)%
\psTransformPoint[RotX=-60](-5 -3 0)(xE,yE,zE){A}
\psTransformPoint[RotX=-60](-5 3 0)(xE,yE,zE){B}
\psTransformPoint[RotX=-60](5 3 0)(xE,yE,zE){C}
\psTransformPoint[RotX=-60](5 -3 0)(xE,yE,zE){D}
\pspolygon[fillstyle=vlines,hatchcolor=yellow!90,hatchwidth=0.02,hatchsep=0.04](A)(B)(C)(D)
%
%
\PointEcran(1,1,1){S1}
\psPoint(1,1,1){s1}
\psline(S1)(V)
\psline[linestyle=dashed](s1)(S1)
%
\PointEcran(1,1,-1){S2}
\psPoint(1,1,-1){s2}
\psline(S2)(V)
\psline[linestyle=dashed](s2)(S2)
%
\PointEcran(-1,1,-1){S3}
\psPoint(-1,1,-1){s3}
\psline(S3)(V)
\psline[linestyle=dashed](s3)(S3)
%
\PointEcran(-1,1,1){S4}
\psPoint(-1,1,1){s4}
\psline(S4)(V)
\psline[linestyle=dashed](s4)(S4)
%
\PointEcran(1,-1,-1){S5}
\psPoint(1,-1,-1){s5}
\psline(S5)(V)
\psline[linestyle=dashed](s5)(S5)
%
\PointEcran(1,-1,1){S6}
\psPoint(1,-1,1){s6}
\psline(S6)(V)
\psline[linestyle=dashed](s6)(S6)
%
\PointEcran(-1,-1,1){S7}
\psPoint(-1,-1,1){s7}
\psline(S7)(V)
\psline[linestyle=dashed](s7)(S7)
\psset{solidmemory}
\psSolid[object=plan,
   definition=equation,
   args={[0 0 1 0]},
   base=-5 5 -3 3,
   RotX=-60,
%   showBase,
   action=none,
   name=planbase,
]
%% here, we define the plantype object "Ecran"
\codejps{
   planbase
   dup xE yE zE planputorigine
   dup -180 rotateplan
   /Ecran exch def
}%
%% uncomment follow line to draw "Ecran"
%\psSolid[object=plan,args=Ecran,showBase,planmarks]
\psProjection[object=texte,
   plan=Ecran,
   fontsize=20,
   text=Projection Screen](-2,2)

%
\psset{linecolor=red,fillstyle=vlines,hatchsep=0.04,hatchwidth=0.02}
\pspolygon[hatchcolor=red!60](S1)(S2)(S3)(S4)
\pspolygon[,hatchcolor=red!60](S1)(S2)(S5)(S6)
\pspolygon[hatchcolor=red!10](S1)(S4)(S7)(S6)
\psdots(s1)(s2)(s3)(s4)(s5)(s6)(s7)(S1)(S2)(S3)(S4)(S5)(S6)(S7)
\psbrace[ref=lC,linecolor=black](V)(E){$D$}
\uput[45](V){View Point}
\end{pspicture}
\end{center}

The coordinates of the object, in this case the bluish cube, are setup in the axes of coordinates $Oxyz$.  The \Index{coordinates} of the \Index{view point} ($V$), are setup in the same axes of coordinates, either in \Index{spherical coordinates}---with the adding option \verb+[rtp2xyz]+, or in Cartesian coordinates---which is the default option.

Example: \verb+[viewpoint=50 30 20  rtp2xyz]+ \qquad (here the notation with spherical coordinates)


See some examples:

\def\decor{%
\psset{solidmemory}
 \psSolid[object=plan,
   definition=equation,
   base=-5 5 -5 5,
   args={[0 0 1 0] 180},
   name=P1]%
\psset{fontsize=28.45,plan=P1}
\psSolid[object=plan,
   args=P1,
   plangrid,action=none]
\psProjection[object=texte,
   linecolor=red,
   text=pst-solides3d](0,3.5)
 \psSolid[object=sphere,r=1,fillcolor=red!25,ngrid=18 36](4,4,1)
 \psSolid[object=cone,h=3,r=1,fillcolor=cyan,mode=5](-4,4,0)
 \psSolid[object=cube,a=2,fillcolor=magenta!20](-4,-4,1)
 \psSolid[object=cylindre,r=1,h=4,fillcolor=blue!20,ngrid=4 16](4,-4,0)
\axesIIID(0,0,0)(6,6,6)
\psPoint(0,0,0){O}
\psdot(O)}

\begin{pspicture}(-3,-3)(3,3)
%\psframe(-5,-3)(4,4)
 \psset{viewpoint=20 25 15,Decran=20,lightsrc=viewpoint,unit=0.9}
\decor
\rput(0,-4){\texttt{viewpoint=20 25 15}}
 \end{pspicture}\qquad\qquad\qquad\qquad
\begin{pspicture}(-3,-3)(3,3)
%\psframe(-5,-3)(4,4)
 \psset{viewpoint=-10 0 30,Decran=20,lightsrc=viewpoint,unit=0.9}
\decor
\rput(0,-4){\texttt{viewpoint=-10 0 30}}
 \end{pspicture}


\begin{pspicture}(-3,-3)(3,4.5)
%\psframe(-5,-3)(4,4)
 \psset{viewpoint=-20 0 10,Decran=10,lightsrc=viewpoint,unit=0.9}
\decor
\rput(0,-4){\texttt{viewpoint=-20 0 10}}
 \end{pspicture}\qquad\qquad\qquad\qquad
 \begin{pspicture}(-3,-3)(3,4.5)
%\psframe(-5,-3)(4,4)
 \psset{viewpoint=-20 -10 25,Decran=20,lightsrc=viewpoint,unit=0.9}
\decor
\rput(0,-4){\texttt{viewpoint=-20 -10 25}}
 \end{pspicture}

\section{The definition of the option \texttt{\Index{Decran}}}
The \Index{projection screen} is placed perpendicular to the direction $OV$---central
perspective, at a distance $D$ from the view point $V$: We call that distance
`Decran', with the default value of \texttt{\Lkeyword{Decran}=50}; this value can
either be positive or negative.



The following examples show the behaviour of the parameter \Lkeyword{Decran}.

\begin{center}
\begin{pspicture}(-2,-3)(2.5,3)
\psaxes[yAxis=false](-2,-2)(2,2)
\psset{viewpoint=0 0 5,Decran=5}
\psSolid[object=grille,base=-2 2 -2 2]
\psSolid[object=vecteur,args=0 0 0  2 2 0,linecolor=red,linewidth=2pt]
\axesIIID(3,3,3)\pnode(2,-2){B}\pnode(2,2){A}
\end{pspicture}
\qquad
\begin{pspicture}(-0.5,-3)(5,3)
\psaxes[yAxis=false](0,-2)(5,2)
\psset{viewpoint=5 0 5,Decran=5,RotX=-90}
\psSolid[object=grille,base=-2 2 -2 2,RotX=89.9]
\axesIIID[axisnames={x,z,y}](3,3,0)
\psdot(5,0)\uput[0](5,0){V}
\psline[tbarsize=3mm 5]{<->|}(0,-0.5)(5,-0.5)\rput*(2.5,-0.5){$D=V$}
\psline[linestyle=dashed](0,2)(5,0)\psline[linestyle=dashed](0,-2)(5,0)
\uput[-90](0,-2.5){Original}\uput[-90](0,-2.85){Image}
\psline[linestyle=dotted](A)(0,2)
\psline[linestyle=dotted](B)(0,-2)
\rput(-1,2.75){Rotation: }
\rput(-1,2.25){90$^\circ$ around $x$}
\psSolid[object=vecteur,args=0 0 0  2 2 0,linecolor=red,linewidth=2pt]
\end{pspicture}\\[\normalbaselineskip]
%
\begin{pspicture}(-2,-3)(2.5,3)
\psaxes[yAxis=false](-2,-2)(2,2)
\psset{viewpoint=0 0 5,Decran=2.5}
\psSolid[object=grille,base=-2 2 -2 2]
\psSolid[object=vecteur,args=0 0 0  2 2 0,linecolor=red,linewidth=2pt]
\axesIIID(3,3,3)\pnode(1,-1){B}\pnode(1,1){A}
\end{pspicture}
\qquad
\begin{pspicture}(-0.5,-3)(5,3)
\psaxes[yAxis=false](0,-2)(5,2)
\psset{viewpoint=5 0 5,Decran=2.5,RotX=-90}
\psline[linewidth=1pt](0,2)(0,-2)
\psline[linewidth=1.5pt,linecolor=red]{->}(0,0)(0,-2)
\psdot(5,0)\uput[0](5,0){V}
\psline[tbarsize=3mm 5]{<->|}(0,1.5)(5,1.5)\rput*(2.5,1.5){$V$}
\psline[linestyle=dashed](0,2)(5,0)\psline[linestyle=dashed](0,-2)(5,0)
\psline[tbarsize=3mm 5]{|<->|}(2.5,-1.5)(5,-1.5)\rput*(3.75,-1.5){$D$}
\psline[linewidth=1.5pt](2.5,1)(2.5,-1)
\psline[linewidth=1.5pt,linecolor=red]{->}(2.5,0)(2.5,-1)
\psline{->}(2.5,0)(3.5,0)\uput[0](3.5,0){$z$}
\uput[-90](0,-2.5){Original}\uput[-90](2.5,-2.5){Image}
\psline[linestyle=dotted](A)(2.5,1)
\psline[linestyle=dotted](B)(2.5,-1)
\rput(-1.5,1.75){Rotation:}
\rput(-1.5,1.25){90$^\circ$ around $x$}
\end{pspicture}
\end{center}


If you keep the view point and make the \Lkeyword{Decran} value smaller, then the
image gets smaller. If you make the \Lkeyword{Decran} value larger, then the image gets larger.

Here are some examples, where we keep the same object, the same view point
and just vary the \Lkeyword{Decran} value:

\begin{center}
\begin{pspicture}(-2,-2)(2,2)
%\psgrid
\psset{solidmemory}
\psset{viewpoint=0 50 0,Decran=50}
%\psSolid[object=sphere,r=2,ngrid=18 36]
\psSolid[object=plan,definition=normalpoint,plangrid,linecolor=red,
         base=-2 2 -2 2,args={0 0 0 [0 1 0 180]},name=monplan]
\psset{plan=monplan}
\psProjection[object=texte,
              linecolor=red,
              fontsize=105.35,
              text=PS]%
\composeSolid
\rput*(0,-1.75){\texttt{Decran=50}}
\end{pspicture}\qquad
\begin{pspicture}(-2,-2)(2,2)
%\psgrid
\psset{solidmemory}
\psset{viewpoint=0 50 0,Decran=25}
%\psSolid[object=sphere,r=2,ngrid=18 36]
\psSolid[object=plan,definition=normalpoint,plangrid,linecolor=red,
         base=-2 2 -2 2,args={0 0 0 [0 1 0 180]},name=monplan]
\psset{plan=monplan}
\psProjection[object=texte,
              linecolor=red,
              fontsize=105.35,
              text=PS]%
\composeSolid
\rput*(0,-1.75){\texttt{Decran=25}}
\end{pspicture}\qquad
\begin{pspicture}(-2,-2)(2,2)
%\psgrid
\psset{solidmemory}
\psset{viewpoint=0 50 0,Decran=-50}
\psSolid[object=plan,definition=normalpoint,plangrid,linecolor=red,
         base=-2 2 -2 2,args={0 0 0 [0 1 0 180]},name=monplan]
\psset{plan=monplan}
\psProjection[object=texte,
              linecolor=red,
              fontsize=105.35,
              text=PS]%
\composeSolid
\rput*(0,-1.75){\texttt{Decran=-50}}
\end{pspicture}
\end{center}


\endinput


