\section{Colours and the nuances of a colour}

The key word \texttt{\Lkeyword{fillcolor}=colourname} allows us to specify the wanted colour for the outer faces of a solid.
The key word \texttt{\texttt{\Lkeyword{incolor}=colourname}} allows us to specify the wanted colour for the inner faces of a solid.

The possible values for \textit{name} are those known to PSTricks (and particularly those of the package \texttt{xcolor}).

We can directly use the colour nuances in the color schemes of
HSB, RGB or CMYK. In that case we use the key values \Lkeyval{hue},
\Lkeyval{inhue} or \Lkeyval{inouthue} for the outer faces, the inner faces, or for all the faces.
The number of arguments \Lkeyval{hue} determines nuances.

\subsection{Predefined \Index{colours} by the option \texttt{dvipsnames}}

There are $68$~predefined \Index{colours}, which are identified by
\texttt{solides.pro}: \texttt{Black}, \texttt{White}, and the
$66$~colours below.

\bgroup\centering
\newcommand{\colorcube}[1]{%
\begin{pspicture}(-1.2,-1)(1.2,1)
\psframe(-1.2,-1)(1.2,1)
\psset{viewpoint=50 50 20 rtp2xyz,Decran=150,lightsrc=viewpoint}
\psSolid[object=datfile,
    file=./cubecolor,
    linewidth=0.07\pslinewidth,
    linecolor=#1!50,
    fillcolor=#1,
    action=draw**]
\rput(0,-0.75){\footnotesize \texttt{#1}}
\end{pspicture}
}

\parindent0pt
%\parskip-8pt
\colorcube{GreenYellow}
\colorcube{Yellow}
\colorcube{Goldenrod}
\colorcube{Dandelion}
\colorcube{Apricot}
\colorcube{Peach}

\colorcube{Melon}
\colorcube{YellowOrange}
\colorcube{Orange}
\colorcube{BurntOrange}
\colorcube{Bittersweet}
\colorcube{RedOrange}

\colorcube{Mahogany}
\colorcube{Maroon}
\colorcube{BrickRed}
\colorcube{Red}
\colorcube{OrangeRed}
\colorcube{RubineRed}

\colorcube{WildStrawberry}
\colorcube{Salmon}
\colorcube{CarnationPink}
\colorcube{Magenta}
\colorcube{VioletRed}
\colorcube{Rhodamine}

\colorcube{Mulberry}
\colorcube{RedViolet}
\colorcube{Fuchsia}
\colorcube{Lavender}
\colorcube{Thistle}
\colorcube{Orchid}

\colorcube{DarkOrchid}
\colorcube{Purple}
\colorcube{Plum}
\colorcube{Violet}
\colorcube{RoyalPurple}
\colorcube{BlueViolet}

\colorcube{Periwinkle}
\colorcube{CadetBlue}
\colorcube{CornflowerBlue}
\colorcube{MidnightBlue}
\colorcube{NavyBlue}
\colorcube{RoyalBlue}

\colorcube{Blue}
\colorcube{Cerulean}
\colorcube{Cyan}
\colorcube{ProcessBlue}
\colorcube{SkyBlue}
\colorcube{Turquoise}

\colorcube{TealBlue}
\colorcube{Aquamarine}
\colorcube{BlueGreen}
\colorcube{Emerald}
\colorcube{JungleGreen}
\colorcube{SeaGreen}

\colorcube{Green}
\colorcube{ForestGreen}
\colorcube{PineGreen}
\colorcube{LimeGreen}
\colorcube{YellowGreen}
\colorcube{SpringGreen}

\colorcube{OliveGreen}
\colorcube{RawSienna}
\colorcube{Sepia}
\colorcube{Brown}
\colorcube{Tan}
\colorcube{Gray}

\egroup

\subsection{Predefined \Index{colours} by the option \texttt{svgnames}}

The following colours are known by PSTricks, when the option \texttt{svgnames} is given.
These ones are not identified by the file \texttt{solides.pro}: we can use them directly with the option \Lkeyword{fcol}.

\bgroup
\newcommand{\colorcone}[1]{%
\begin{pspicture}(-1.2,-1)(1.2,1)
\psframe(-1.2,-1)(1.2,1)
\psset{viewpoint=50 50 20 rtp2xyz,Decran=150,lightsrc=viewpoint}
\psSolid[object=cone,
    linewidth=0.07\pslinewidth,
%    linecolor=#1!50,
    fillcolor=#1,
    ngrid=4 12,
    r=0.2,h=0.37,
    action=draw**](0,0,-0.05)
\rput(0,-0.75){\footnotesize \texttt{#1}}
\end{pspicture}
}


\parindent0pt
%\parskip-8pt

These colours are delivered from the package \texttt{xcolor}.
\bigskip

{\centering
\colorcone{AliceBlue}
\colorcone{AntiqueWhite}
\colorcone{Aqua}
\colorcone{Aquamarine}
\colorcone{Azure}
\colorcone{Beige}

\colorcone{Bisque}
\colorcone{Black}
\colorcone{BlanchedAlmond}
\colorcone{Blue}
\colorcone{BlueViolet}
\colorcone{Brown}

\colorcone{BurlyWood}
\colorcone{CadetBlue}
\colorcone{Chartreuse}
\colorcone{Chocolate}
\colorcone{Coral}
\colorcone{CornflowerBlue}

\colorcone{Cornsilk}
\colorcone{Crimson}
\colorcone{Cyan}
\colorcone{DarkBlue}
\colorcone{DarkCyan}
\colorcone{DarkGoldenrod}

\colorcone{DarkGray}
\colorcone{DarkGreen}
\colorcone{DarkGrey}
\colorcone{DarkKhaki}
\colorcone{DarkMagenta}
\colorcone{DarkOliveGreen}

\colorcone{DarkOrange}
\colorcone{DarkOrchid}
\colorcone{DarkRed}
\colorcone{DarkSalmon}
\colorcone{DarkSeaGreen}
\colorcone{DarkSlateBlue}

\colorcone{DarkSlateGray}
\colorcone{DarkSlateGrey}
\colorcone{DarkTurquoise}
\colorcone{DarkViolet}
\colorcone{DeepPink}
\colorcone{DeepSkyBlue}

\colorcone{DimGray}
\colorcone{DimGrey}
\colorcone{DodgerBlue}
\colorcone{FireBrick}
\colorcone{FloralWhite}
\colorcone{ForestGreen}

\colorcone{Fuchsia}
\colorcone{Gainsboro}
\colorcone{GhostWhite}
\colorcone{Gold}
\colorcone{Goldenrod}
\colorcone{Gray}

\colorcone{Grey}
\colorcone{Green}
\colorcone{GreenYellow}
\colorcone{Honeydew}
\colorcone{HotPink}
\colorcone{IndianRed}

\colorcone{Indigo}
\colorcone{Ivory}
\colorcone{Khaki}
\colorcone{Lavender}
\colorcone{LavenderBlush}
\colorcone{LawnGreen}

\colorcone{LemonChiffon}
\colorcone{LightBlue}
\colorcone{LightCoral}
\colorcone{LightCyan}
\colorcone{LightGoldenrodYellow}
\colorcone{LightGray}

\colorcone{LightGreen}
\colorcone{LightGrey}
\colorcone{LightPink}
\colorcone{LightSalmon}
\colorcone{LightSeaGreen}
\colorcone{LightSkyBlue}

\colorcone{LightSlateGray}
\colorcone{LightSlateGrey}
\colorcone{LightSteelBlue}
\colorcone{LightYellow}
\colorcone{Lime}
\colorcone{LimeGreen}

\colorcone{Linen}
\colorcone{Magenta}
\colorcone{Maroon}
\colorcone{MediumAquamarine}
\colorcone{MediumBlue}
\colorcone{MediumOrchid}

\colorcone{MediumPurple}
\colorcone{MediumSeaGreen}
\colorcone{MediumSlateBlue}
\colorcone{MediumSpringGreen}
\colorcone{MediumTurquoise}
\colorcone{MediumVioletRed}

\colorcone{MidnightBlue}
\colorcone{MintCream}
\colorcone{MistyRose}
\colorcone{Moccasin}
\colorcone{NavajoWhite}
\colorcone{Navy}

\colorcone{OldLace}
\colorcone{Olive}
\colorcone{OliveDrab}
\colorcone{Orange}
\colorcone{OrangeRed}
\colorcone{Orchid}

\colorcone{PaleGoldenrod}
\colorcone{PaleGreen}
\colorcone{PaleTurquoise}
\colorcone{PaleVioletRed}
\colorcone{PapayaWhip}
\colorcone{PeachPuff}

\colorcone{Peru}
\colorcone{Pink}
\colorcone{Plum}
\colorcone{PowderBlue}
\colorcone{Purple}
\colorcone{Red}

\colorcone{RosyBrown}
\colorcone{RoyalBlue}
\colorcone{SaddleBrown}
\colorcone{Salmon}
\colorcone{SandyBrown}
\colorcone{SeaGreen}

\colorcone{Seashell}
\colorcone{Sienna}
\colorcone{Silver}
\colorcone{SkyBlue}
\colorcone{SlateBlue}
\colorcone{SlateGray}

\colorcone{SlateGrey}
\colorcone{Snow}
\colorcone{SpringGreen}
\colorcone{SteelBlue}
\colorcone{Tan}
\colorcone{Teal}

\colorcone{Thistle}
\colorcone{Tomato}
\colorcone{Turquoise}
\colorcone{Violet}
\colorcone{Wheat}
\colorcone{White}

\colorcone{WhiteSmoke}
\colorcone{Yellow}
\colorcone{YellowGreen}

}
\egroup

\subsection{Nuances in the \Index{colour scheme} of \Index{HSB}, \Index{saturation} and maximum \Index{brilliance}}

There are 2 key values: \texttt{\Lkeyword{hue}=$h_0$ $h_1$} where
the numbers $h_0$ and $h_1$ with $0\leq h_0 < h_1 \leq 1$
respect the limits of the colour scheme of HSB.



\psset{viewpoint=50 50 20 rtp2xyz,Decran=30}
\begin{LTXexample}[width=7.5cm]
\psset{unit=1}
\begin{pspicture}(-4,-1.5)(3,1)
\psSolid[object=grille,
   base=-3 5 -3 3,
   linecolor=gray,
   hue=0 1](0,0,0)
\end{pspicture}
\end{LTXexample}



\psset{viewpoint=50 50 20 rtp2xyz,Decran=30}
\begin{LTXexample}[width=7.5cm]
\psset{unit=1}
\begin{pspicture}(-4,-1.5)(3,1)
\psSolid[object=grille,
   base=-3 5 -3 3,
   linecolor=gray,
   hue=0 .3](0,0,0)
\end{pspicture}
\end{LTXexample}



\psset{viewpoint=50 50 20 rtp2xyz,Decran=30}
\begin{LTXexample}[width=7.5cm]
\psset{unit=1}
\begin{pspicture}(-4,-1.5)(3,1)
\psSolid[object=grille,
   base=-3 5 -3 3,
   linecolor=gray,
   hue=.5 .6](0,0,0)
\end{pspicture}
\end{LTXexample}


\subsection{Nuances in the \Index{colour scheme} of \Index{HSB}, \Index{saturation} and fixed \Index{brilliance}}

There are 4 key values: \texttt{\Lkeyword{hue}=$h_0$ $h_1$ $s$ $b$} or
the numbers $h_0$ and $h_1$ with $0\leq h_0 < h_1 \leq 1$
respect the limits of the colour scheme HSB and $s$
and $b$ are the values for \texttt{saturation} and \texttt{brillance}.

\psset{viewpoint=50 50 20 rtp2xyz,Decran=30}
\begin{LTXexample}[width=7.5cm]
\psset{unit=1}
\begin{pspicture}(-4,-1.5)(3,1)
\psSolid[object=grille,
   base=-3 5 -3 3,
   linecolor=gray,
   hue=0 1 .8 .7](0,0,0)
\end{pspicture}
\end{LTXexample}




\psset{viewpoint=50 50 20 rtp2xyz,Decran=30}
\begin{LTXexample}[width=7.5cm]
\psset{unit=1}
\begin{pspicture}(-4,-1.5)(3,1)
\psSolid[object=grille,
   base=-3 5 -3 3,
   linecolor=gray,
   hue=0 1 .5 1](0,0,0)
\end{pspicture}
\end{LTXexample}

\subsection{Nuances in the \Index{colour scheme} of \Index{HSB}, gneral case}

There are 7 key values: \texttt{\Lkeyword{hue}=$h_0$ $s_0$ $b_0$ $h_1$ $s_1$
$b_1$ (hsb)} or the numbers $h_i$, $s_i$ and $b_i$ respecting the limits of the parameters of HSB.



\psset{viewpoint=50 50 20 rtp2xyz,Decran=30}
\begin{LTXexample}[width=7.5cm]
\psset{unit=1}
\begin{pspicture}(-4,-1.5)(3,1)
\psSolid[object=grille,
   base=-3 5 -3 3,
   linecolor=gray,
   hue=0 .8 1 1 1 .7 (hsb)](0,0,0)
\end{pspicture}
\end{LTXexample}

\subsection{Nuances in the \Index{colour scheme} of \Index{RGB}}

There are 6 key values: \texttt{\Lkeyword{hue}=$r_0$ $g_0$ $b_0$ $r_1$ $g_1$
$b_1$} or the numbers $r_i$, $g_i$ and $b_i$ respecting the limits of the $3$ parameters of RGB.



\psset{viewpoint=50 50 20 rtp2xyz,Decran=30}
\begin{LTXexample}[width=7.5cm]
\psset{unit=1}
\begin{pspicture}(-4,-1.5)(3,1)
\psSolid[object=grille,
   base=-3 5 -3 3,
   linecolor=gray,
   hue=1 0 0 0 0 1](0,0,0)
\end{pspicture}
\end{LTXexample}


\subsection{Nuances in the \Index{colour scheme} of \Index{CMYK}}

There are 8 key values: \texttt{\Lkeyword{hue}=$c_0$ $m_0$ $y_0$ $k_0$ $c_1$ $m_1$
$y_1$ $k_1$} or the numbers $c_i$, $m_i$, $y_i$ and $k_i$ respecting the limits of the $4$ parameters of CMYK.



\psset{viewpoint=50 50 20 rtp2xyz,Decran=30}
\begin{LTXexample}[width=7.5cm]
\psset{unit=1}
\begin{pspicture}(-4,-1.5)(3,1)
\psSolid[object=grille,
   base=-3 5 -3 3,
   linecolor=gray,
   hue=1 0 0 0 0 0 1 0](0,0,0)
\end{pspicture}
\end{LTXexample}

\subsection{Nuances between 2 named colours}

There are 2 key values
\texttt{\Lkeyword{hue}=(color1) (color2)} where
\texttt{color1} and \texttt{color2} are the names of colours known by \verb+solides.pro+.



\psset{viewpoint=50 50 20 rtp2xyz,Decran=30}
\begin{LTXexample}[width=7.5cm]
\psset{unit=1}
\begin{pspicture}(-4,-1.5)(3,1)
\psSolid[object=grille,
   base=-3 5 -3 3,
   linecolor=gray,
   hue=(jaune) (CadetBlue)](0,0,0)
\end{pspicture}
\end{LTXexample}

If we like to use some defined colours of \texttt{xcolor}, we use the
key values \texttt{color1}, \texttt{color2}, etc. from \Lcs{psSolid}.

\psset{viewpoint=50 50 20 rtp2xyz,Decran=30}
\begin{LTXexample}[width=7.5cm]
\psset{unit=1}
\begin{pspicture}(-4,-1.5)(3,1)
\psSolid[object=grille,
   base=-3 5 -3 3,
   linecolor=gray,
   color1=red!50,
   color2=green!20,
   hue=(color1) (color2)](0,0,0)
\end{pspicture}
\end{LTXexample}

\subsection{Deactivation of the colour application}
For specific purposes it is possible to disable the application of colour.
This is particularly the case, when an  object is already memorized or defined in external files.
 Within these configurations, if we do not deactivate the colours and
 if we do not define some new colours, these will be the colours by default that overwrite the colours that were defined.


To deactivate the colour application we use the option
\Lkeyword{deactivatecolor}.

\endinput
