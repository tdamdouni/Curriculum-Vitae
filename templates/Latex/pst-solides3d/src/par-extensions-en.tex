\section{Creating your own object}
It is possible to create your own object in a separate file and
import it into the list of objects recognized by
\texttt{pst-solides3d}. Create a text file with the extension of \texttt{.pro}
(myObj.pro) and enter the PostScript commands to define your
\texttt{pst-solides3d} object.

Reference your \texttt{.pro} file in the preamble with
\begin{verbatim}
    \pstheader{myObj.pro}
\end{verbatim}
Following this line, add this new object to the list of objects recognized by \texttt{pst-solides3d}
with
\begin{verbatim}
    \addtosolideslistobject{myObj}
\end{verbatim}

For some examples of this technique, see the following web pages:

\centerline{\url{http://melusine.eu.org/syracuse/mluque/solides3d2007/cristaux/}}

\centerline{\url{http://melusine.eu.org/syracuse/mluque/solides3d2007/rhombicuboctaedre/}}


\section{Creating a \texttt{.u3d} file}

You can manipulate 3D objects created with \texttt{pst-solides3d};
the following three steps are necessary:
\begin{enumerate}
\item Save your designed 3D object in the \texttt{.off} or
    \texttt{.obj} format---see the chapter ``\textit{Usage of external files}''.

\item Then use, for example, \textit{Meshlab}---an open source software---(\url{http://meshlab.sourceforge.net/}) to convert these files
    into the \texttt{.u3d} format.

\item The {\LaTeX} package \texttt{movie15} of Alexander \textsc{Grahn} embeds
    files in the \texttt{.u3d} format into a PDF document, the document can then be viewed
    using $\text{Adobe}^{\text{\tiny\circledR}}$ $\text{Reader}^{\text{\tiny\circledR}}$ 7 or later.
\end{enumerate}

You will find some examples on the following web pages:

\centerline{\url{http://melusine.eu.org/syracuse/mluque/solides3d2007/pdf3d/}}

\centerline{\url{http://melusine.eu.org/syracuse/mluque/solides3d2007/zeolithes/}}

\endinput
