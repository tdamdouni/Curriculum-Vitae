\section{Emptying a solid}
Several of the predefined solids have a ``\textit{hollow}'' relative which is naturally associated with it (the cone, the truncated cone, the cylinder, the prism and the spherical zone). For all those, the option \texttt{\Lkeyword{hollow}=true} is provided.
Set to \texttt{false}, we get the ``filled'' solid; set to \texttt{true} we get the ``hollow'' version.


\subsubsection{Example 1: a \Index{cylinder} and a \Index{hollow cylinder}}



\begin{LTXexample}[width=5cm]
\psset{unit=0.5}
\psset{lightsrc=viewpoint,viewpoint=50 60 25 rtp2xyz,Decran=50}
\begin{pspicture}(-2,-3)(6,6)
\psSolid[object=cylindre,h=6,r=2,
   fillcolor=yellow,
      ](0,4,0)
\end{pspicture}
\end{LTXexample}

\begin{LTXexample}[width=5cm]
\psset{unit=0.5}
\psset{lightsrc=viewpoint,viewpoint=50 60 25 rtp2xyz,Decran=50}
\begin{pspicture}(-2,-3)(6,6)
\psSolid[object=cylindre,h=6,r=2,
   fillcolor=yellow,incolor=red,
   hollow](0,4,0)
\end{pspicture}
\end{LTXexample}


\newpage

\subsubsection{Example 2: a \Index{prism} and a \Index{hollow prism}}

\begin{LTXexample}[width=8.7cm]
\psset{unit=0.5}
\psset{lightsrc=viewpoint,viewpoint=50 60 25 rtp2xyz,Decran=50}
\begin{pspicture}(-9,-4)(4,8)
\defFunction{F}(t){t cos 3 mul}{t sin 3 mul}{}
\defFunction{G}(t){t cos}{t sin}{}
\psSolid[object=grille,base=-6 6 -4 4,action=draw]%
\psSolid[object=prisme,
    h=8,fillcolor=yellow,
    RotX=90,ngrid=8 18,
    base=0 180 {F} CourbeR2+
         180 0 {G} CourbeR2+](0,4,0)
\axesIIID(3,4,3)(8,6,7)
\end{pspicture}
\end{LTXexample}

\begin{LTXexample}[width=8.7cm]
\psset{unit=0.5}
\psset{lightsrc=viewpoint,viewpoint=50 60 25 rtp2xyz,Decran=50}
\begin{pspicture}(-9,-4)(3,8)
\defFunction{F}(t){t cos 3 mul}{t sin 3 mul}{}
\defFunction{G}(t){t cos}{t sin}{}
\psSolid[object=grille,base=-6 6 -4 4,action=draw]%
\psSolid[object=prisme,
    h=8,fillcolor=yellow,incolor=red,
    RotX=90,hollow,ngrid=8 18,
    base=0 180 {F} CourbeR2+
         180 0 {G} CourbeR2+](0,4,0)
\axesIIID(3,4,3)(8,6,7)
\end{pspicture}
\end{LTXexample}

\newpage
\subsubsection{Example 3: a \Index{spherical zone} and a \Index{hollow spherical zone}}

\begin{LTXexample}[width=7.5cm]
\psset{unit=0.5}
\psset{lightsrc=10 20 30,viewpoint=50 60 25 rtp2xyz,Decran=50}
\begin{pspicture}(-7,-4)(5,7)
\psSolid[object=grille,
    base=-5 5 -5 5,
    action=draw]%
\psSolid[object=calottesphere,
    r=3,ngrid=16 18,
    fillcolor=cyan!50,
    incolor=yellow,
    theta=45,phi=-30](0,0,1.5)%
\axesIIID(3,3,3.6)(6,6,5)
\end{pspicture}
\end{LTXexample}

\begin{LTXexample}[width=7.5cm]
\psset{unit=0.5}
\psset{lightsrc=10 20 30,viewpoint=50 60 25 rtp2xyz,Decran=50}
\begin{pspicture}(-7,-5)(7,5)
\psSolid[object=calottesphere,
    r=3,ngrid=16 18,
    fillcolor=cyan!50,
    incolor=yellow,
    theta=45,phi=-30,
    hollow,
    RotY=-80]%
\axesIIID(0,3,3)(6,5,4)
\end{pspicture}
\end{LTXexample}


\endinput
