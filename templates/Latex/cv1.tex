\documentclass[12pt,a4paper]{scrartcl}
\usepackage[T1]{fontenc}
\usepackage[latin1]{inputenc}
\usepackage{%
  ngerman,
  ae,
  times,
  %picins, 
  graphicx,
  hyperref,
  currvita}
 
%% Hier schreibt man die PDF Eigenschaften hinein:
\hypersetup{
  pdftitle={Lebenslauf von Maxi Muster}, %%
  pdfauthor={Maxi Muster}, %%
  pdfsubject={}, %%
  pdfcreator={LaTeX2e and pdfLaTeX with hyperref-package.},  
  pdfproducer={}, %%
 
 
  pdfkeywords={} %%
}
 
\newcommand*{\ac}[1]{\mbox{#1}}
\tolerance=600
 
\begin{document}
 
\thispagestyle{empty}
%% Hier kann man ein Bewerbungsportrait oder eine andere 
%% Grafik ins Dokument einbinden. Einfach die naechste 
%% Zeile auskommentieren:
% \parpic[r]{\includegraphics[width=0.25\textwidth]{meinBewerbungsfoto}}
 
%% Hier folgt der Lebenslauf. Einfach an die eigenen Beduerfnisse 
%% anpassen:
\begin{cv}{Lebenslauf}
\vspace{0.5cm}
  \begin{cvlist}{Maxi Muster}
  \item[Adresse:] Muster~Str.~20\\12345 Musterstadt
  \item[Telefon:] (01\,23)~4\,56\,78\,90
%  \item[Mobil:] (01\,23)~4\,56\,78\,90
  \item[E-Mail:] maxi@muster.de
%  \item[WWW:] http://www.
  \item[Geboren~am:]01. Juli 1977 
  \item[Ort:] Musterstadt
  \item[Familienstand:] Ledig
  \item[Nationalit"at] Deutsch
  \end{cvlist}
 
  \begin{cvlist}{Schule \& Studium}
  \item[1994 -- 1997] Abitur in Musterstadt
  \item[1997 -- 2004] Studium der Politik und Philosophie, 
Universit"at Musterstadt
  \item[Januar 2004] Magister der Politischen Wissenschaft mit der 
Note \textit{sehr gut}
  \end{cvlist}
 
  \begin{cvlist}{Berufspraxis}
  \item[2000 -- 2003] Call Center Agent f"ur Musterfabrik
  \item[2004 -- 2005] Praktikum bei der Allgemeinen Musterzeitung
  \end{cvlist}
 
  \begin{cvlist}{Fremdsprachen}
  \item[] Sehr gute Englisch-Kenntnisse in Wort und Schrift.
  \item[] Grundkenntnisse in Franz"osisch
  \end{cvlist}
 
  \begin{cvlist}{Kenntnisse \& Interessen}
  \item[] Sehr guter Umgang mit \LaTeX und HTML.
  \end{cvlist}
 
  \begin{cvlist}{Besondere F"ahigkeiten}
  \item[] Spa"s am Kontakt und Austausch mit Menschen.
  \item[] Interesse am politischen Tagesgeschehen.
  \end{cvlist}
 
  \cvplace{Musterstadt}
  \date{den \today}
\end{cv}
\end{document}